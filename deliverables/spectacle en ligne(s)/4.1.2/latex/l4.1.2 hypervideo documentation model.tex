\documentclass[runningheads,a4paper]{llncs}

\usepackage[utf8]{inputenc}

\usepackage{amssymb}
\setcounter{tocdepth}{3}
\usepackage{graphicx}

\usepackage{url}
\newcommand{\keywords}[1]{\par\addvspace\baselineskip
\noindent\keywordname\enspace\ignorespaces#1}

\begin{document}

\mainmatter  % start of an individual contribution

% first the title is needed
\title{Projet ``Spectacle en ligne(s)''---Livrable L4.1.2\\``Modèles documentaires pour les hypervidéos sur le web''}

% a short form should be given in case it is too long for the running head
\titlerunning{Spectacle en ligne(s)---L4.1.2}

% the name(s) of the author(s) follow(s) next
\author{
  Thomas Steiner\textsuperscript{1} \and
  Pierre-Antoine Champin\textsuperscript{1} \and \\
  Benoît Encelle\textsuperscript{1}\and
  Yannick Prié\textsuperscript{2}
}
%
\authorrunning{Steiner, Encelle, Champin and Prié}
% (feature abused for this document to repeat the title also on left hand pages)

% the affiliations are given next
\institute{
  \textsuperscript{1}CNRS, Université de Lyon, LIRIS -- UMR5205, Université Lyon~1, France\\
  \email{\{tsteiner, pierre-antoine.champin\}@liris.cnrs.fr, benoit.encelle@univ-lyon1.fr}\\
  \textsuperscript{2}CNRS, Université de Nantes, LINA -- UMR 6241, France\\
  \email{yannick.prie@univ-nantes.fr}
}

\maketitle


\begin{abstract}
\keywords{Annotation, documentation model, hypervideo}
\end{abstract}


\section{Introduction}

The term \emph{hypervideo} is commonly used to refer to
``a~displayed video stream that contains embedded user-clickable anchors''%
~\cite{sawhney1996hypercafe,smith2002extensible}.
In a~2006 article in The Economist, the authors write 
``[h]yperlinking video involves the use of ``object-tracking'' software
to make filmed objects, such as cars, clickable as they move around.
Viewers can then click on items of interest in a~video
to watch a related clip; after it has played,
the original video resumes where it left off.
To inform viewers that a~video is hyperlinked,
editors can add highlights to moving images, use beeps as audible cues,
or display still images from hyperlinked videos
next to the clip that is currently playing''~\cite{economist2006hypervideo}.
In standard literature, hypervideo is considered a~logical consequence
of the related concept of \emph{hypertext}.

In this deliverable, we will look at
documentation models for hypervideos on the Web.


\bibliographystyle{abbrv}
\bibliography{references}
\end{document}
