\documentclass{sig-alternate}

\usepackage[utf8]{inputenc}
\usepackage{enumitem}
\usepackage[hyphens]{url}
\usepackage[pdftex,urlcolor=black,colorlinks=true,linkcolor=black,citecolor=black]{hyperref}
\def\sectionautorefname{Section}
\def\subsectionautorefname{Subsection}

% listings and Verbatim environment
\usepackage{fancyvrb}
\usepackage{relsize}
\usepackage{listings}
\usepackage{verbatim}
\newcommand{\defaultlistingsize}{\fontsize{8pt}{9.5pt}}
\newcommand{\inlinelistingsize}{\fontsize{8pt}{11pt}}
\newcommand{\smalllistingsize}{\fontsize{7.5pt}{9.5pt}}
\newcommand{\listingsize}{\defaultlistingsize}
\RecustomVerbatimCommand{\Verb}{Verb}{fontsize=\inlinelistingsize}
\RecustomVerbatimEnvironment{Verbatim}{Verbatim}{fontsize=\defaultlistingsize}
\lstset{frame=lines,captionpos=b,numberbychapter=false,escapechar=§,
        aboveskip=2em,belowskip=1em,abovecaptionskip=0.5em,belowcaptionskip=0.5em,
        framexbottommargin=-1em,basicstyle=\ttfamily\listingsize\selectfont}

% use Courier from this point onward
\let\oldttdefault\ttdefault
\renewcommand{\ttdefault}{pcr}
\let\oldurl\url
\renewcommand{\url}[1]{\inlinelistingsize\oldurl{#1}}

\lstdefinelanguage{JavaScript}{
  keywords={push, typeof, new, true, false, catch, function, return, null,
    catch, switch, var, if, in, while, do, else, case, break},
  keywordstyle=\bfseries,
  ndkeywords={class, export, boolean, throw, implements, import, this},
  ndkeywordstyle=\color{darkgray}\bfseries,
  identifierstyle=\color{black},
  sensitive=false,
  comment=[l]{//},
  morecomment=[s]{/*}{*/},
  commentstyle=\color{darkgray},
  stringstyle=\color{red},
  morestring=[b]',
  morestring=[b]"
}

% linewrap symbol
\usepackage{color}
\definecolor{grey}{RGB}{130,130,130}
\newcommand{\linewrap}{\raisebox{-.6ex}{\textcolor{grey}{$\hookleftarrow$}}}

% todo macro
\usepackage{color}
\newcommand{\todo}[1]{\noindent\textcolor{red}{{\bf \{TODO}: #1{\bf \}}}}

\hyphenation{WebVTT}

\newcommand{\vtt}[1]{\texttt{vtt:#1}}

\begin{document}
%
% --- Author Metadata here ---
\conferenceinfo{International World Wide Web Conference}{2014 Seoul, Korea}
\CopyrightYear{2014} % Allows default copyright year (20XX) to be over-ridden - IF NEED BE.
%\crdata{0-12345-67-8/90/01}  % Allows default copyright data (0-89791-88-6/97/05) to be over-ridden - IF NEED BE.
% --- End of Author Metadata ---

\title{Weaving the Web(VTT) of Data}

\numberofauthors{6}

\author{
% 1st. author
\alignauthor
Thomas Steiner\titlenote{Second affiliation: \emph{Google Germany GmbH, Hamburg, DE}}\\
       \affaddr{CNRS, Université de Lyon}
       \affaddr{LIRIS, UMR5205}\\
       \affaddr{Université Lyon~1, France}\\
       \email{\fontsize{12pt}{14.4pt}\sffamily\selectfont tsteiner@liris.cnrs.fr}
% 2nd. author
\alignauthor
Pierre-Antoine Champin\\
       \affaddr{CNRS, Université de Lyon}
       \affaddr{LIRIS, UMR5205}\\
       \affaddr{Université Lyon~1, France}\\
       \email{\fontsize{12pt}{14.4pt}\sffamily\selectfont pachampin@liris.cnrs.fr}
% 3rd. author
\alignauthor
Benoît Encelle\\
       \affaddr{CNRS, Université de Lyon}
       \affaddr{LIRIS, UMR5205}\\
       \affaddr{Université Lyon~1, France}\\
       \email{\fontsize{12pt}{14.4pt}\sffamily\selectfont bencelle@liris.cnrs.fr}
\and  % use '\and' if you need 'another row' of author names
% 4th. author
\alignauthor
Yannick Prié\\
       \affaddr{Université de Nantes}\\
       \affaddr{LINA -- UMR 6241 CNRS}\\
       \email{\fontsize{12pt}{14.4pt}\sffamily\selectfont yannick.prie@univ-nantes.fr}
}

\maketitle
\begin{abstract}
We propose that WebVTT data can be considered as a rich, and yet still untapped, source of video-related linked data. Indeed, Web standards such as Media Fragments and the Media Annotation ontology make it quite straightforward to rephrase WebVTT as linked data. We present in this paper the principles of this transformation. Then we provide evidence, based on the Common Crawl corpus, of the huge potential of WebVTT as a source of linked data. Finally, we propose a prototype, based on modern Web technology (HTML5 and JSON-LD), easing the consumption \emph{and} production of WebVTT-encoded linked data.

\end{abstract}

\category{H.5.1}{Multimedia Information Systems}{Video}

%\terms{Experimentation, Design, Standardization}

\keywords{WebVTT, Linked Data, Web of Data}

\section{Introduction}

\cite{pfeiffer2013webvtt}

\section{WebVTT as linked data}

The WebVTT specification defines a syntax for conveying video text tracks,
and a semantics for this syntax in terms of how web browser should process such tracks.
It achieves that by specifying an underlying data model for those tracks.
The aim of this section is to show
how this data model can easily be mapped to RDF-based linked data,
thus allowing many other uses of this data.
For this purpose, we propose an RDF-Schema ontology\footnote{
  \url{http://champin.net/2014/linkedvtt/onto\#}
} conveying the WebVTT data model.
In the rest of the paper, terms of this ontology will be preceded by the \vtt{} prefix.
An online implementation of this interpretation process is also available\footnote{
  \url{http://champin.net/2014/linkedvtt/}
}.

\subsection{Basic interpretation}

A WebVTT file defines a set of cues,
which are described by a pair of timestamps and a payload.
In other word, each cue is an annotation of the video,
associating a fragment, delimited by the two timestamps, to the payload.
As we have a standard way of identifying that fragment with a URI~\cite{troncy2012mediafragments}
it is straightfoward to represent this annotation as an RDF triple.
We therefore propose a property \vtt{annotatedBy} to serve as predicate for those triples.

To keep the context of each annotation,
we use the notion of RDF dataset~\cite{cyganiak2014rdf11concepts}.
Each \vtt{annotatedBy} triple is enclosed in a named graph,
whose name is either a URI, based on the cue identifier it has one,
or a blank node if the cue has no identifier.
The default graph of the dataset describes its overall structure,
linking the dataset URI to all the URIs and blank nodes identifying its cues
with the \vtt{hasCue} property.
Each cue is also linked, in the default graph, to the media fragment URI it describes,
with the \vtt{describesFragment} property. 

\todo{illustrate with a figure, representing the WebVTT example that will probably be there in the intro?}

As the notion of dataset is a recent addition to the RDF core concepts
(previously, it was specific to the SPARQL query language),
we envision that some consumers will not be able to deal with it.
Hence we propose an alternate interpretation of WebVTT into RDF.
In this so-called \emph{flat} interpretation,
the content of all named graph is merged into the default graph.

\subsection{Advanced interpretation}

WebVTT is not limited to textual caption tracks.
The HTML5 \textt{track} tag supports indeed different kinds of track,
one of them being \textit{metadata},
a track designed for machine rather than human consumption.
Although there is few evidence of use of this kind of tracks (see next section),
which is understandable as the technology is still under development,
we propose that JSON data is a good candidate for cues of such tracks.
Indeed, JSON has a textual syntax which is easy to author
and easy to process in a web browser.
Furthermore, JSON-LD~\cite{sporny2013jsonld} provides
a standard way to interpret JSON data as linked data,
which fits nicely with our approach.

More precisely, whenever the payload of a cue parses to a JSON object,
we consider that this object is meant to represent the annotated media fragment itself,
and we interpret it as JSON-LD.
Hence, all the property of the JSON object are applied directly to the fragment,
and embedded structures can be used to describe other resources related to that fragment
(\textit{e.g.} depicted persons).
In that case, all the triples generated from parsing the payload as JSON-LD
\emph{replace} the \vtt{annotatedBy} triple in the cue's named graph.
Figure~\{fig:json-payload} gives an example of such a JSON payload.

\begin{figure}
\begin{verbatim}
00:00:00.000 --> 00:00:12.000
{
  "label": "Vince and X in the car",
  "depicts": [
    { "@id": "",
      "name": "" },
    { "@id": "",
      "name": "" }
  ]
}
\end{verbatim}
\label{fig:json-payload}
\caption{A WevVTT cue with a JSON payload}
\end{figure}

Note that JSON-LD requires a \emph{context} to interpret JSON data as linked data,
especially to disambiguate keywords to full URIs.
This context can of course be provided in each cue,
but we provide a convenient way to declare it once for the entier WebVTT file (see below).

\subsection{Linked-data related metadata}

WebVTT files can contain, in addition to the cues, some metadata described as key-value pairs.
While the WebVTT specification defines a number of metadata keys,
it leaves it open for extensions.
We propose three extended metadata keys:

\begin{description}

\item[@base]
Sets the base URI used for resolving relative URIs.
This applies to any relative URIs that would be found in the JSON-LD descriptions,
but also to generate URIs for cues based on their identifiers.
It defaults to the URI of the WebVTT file.

\item[@context]
This key can be used multiple times; each value is the URI of a JSON-LD context
that should be used to interpret the JSON payloads in the WebVTT file.

\item[@video]
Sets the URI for the video for generating media fragment URIs.
If not present, the video URI must be provided externally,
\textit{e.g.} the \texttt{src} attribute of the \texttt{video} tag
containing the WebVTT track.

\end{description}

Of course, WebVTT in the wild does not contain those metadata keys,
but we believe that they allow an easy transition
from plain WebVTT to linked-data WebVTT,
just like JSON-LD makes it easy to turn plain JSON into linked data
by simply adding a \texttt{@context} key.



\section{WebVTT in the wild}

\section{HTML5 prototype (TODO better title)}

\section{Related Work}

\todo{cite LDOW paper on the use of micro-data, as this is a similar approach}

\section{Future Work and Conclusions}
\label{sec:future-work-conclusions}

\section*{Acknowledgments}
\footnotesize
The research presented in this paper was partially supported
by the French National Agency for Research  project
\emph{Spectacle En Ligne(s)}, project reference
\mbox{ANR-12-CORP-0015}.

\normalsize
\bibliographystyle{abbrv}
\bibliography{references}

\end{document}
