\documentclass[runningheads,a4paper]{llncs}

\usepackage[utf8]{inputenc}

\usepackage{amssymb}
\setcounter{tocdepth}{3}
\usepackage{graphicx}

\newcommand{\keywords}[1]{\par\addvspace\baselineskip
\noindent\keywordname\enspace\ignorespaces#1}

\usepackage{pifont} 
\usepackage[utf8]{inputenc}
\usepackage{enumitem}
\usepackage[hyphens]{url}
\usepackage[pdftex,urlcolor=black,colorlinks=true,linkcolor=black,citecolor=black]{hyperref}
\def\sectionautorefname{Section}
\def\subsectionautorefname{Subsection}

% listings and Verbatim environment
\usepackage{fancyvrb}
\usepackage{relsize}
\usepackage{listings}
\usepackage{verbatim}
\newcommand{\defaultlistingsize}{\fontsize{8pt}{9.5pt}}
\newcommand{\inlinelistingsize}{\fontsize{8pt}{11pt}}
\newcommand{\smalllistingsize}{\fontsize{7.5pt}{9.5pt}}
\newcommand{\listingsize}{\defaultlistingsize}
\RecustomVerbatimCommand{\Verb}{Verb}{fontsize=\inlinelistingsize}
\RecustomVerbatimEnvironment{Verbatim}{Verbatim}{fontsize=\defaultlistingsize}
\lstset{frame=lines,captionpos=b,numberbychapter=false,escapechar=§,
        aboveskip=2em,belowskip=1em,abovecaptionskip=0.5em,belowcaptionskip=0.5em,
        framexbottommargin=-1em,basicstyle=\ttfamily\listingsize\selectfont}

% use Courier from this point onward
\let\oldttdefault\ttdefault
\renewcommand{\ttdefault}{pcr}
\let\oldurl\url
\renewcommand{\url}[1]{\inlinelistingsize\oldurl{#1}}

\usepackage[usenames,dvipsnames,svgnames,table]{xcolor}
\lstdefinelanguage{JavaScript}{
  keywords={push, typeof, new, true, false, catch, function, return, null,
    catch, switch, var, if, in, while, do, else, case, break, div, script, video},
  keywordstyle=\bfseries,
  ndkeywords={class, export, boolean, throw, implements, import, this},
  ndkeywordstyle=\color{darkgray}\bfseries,
  identifierstyle=\color{black},
  sensitive=false,
  comment=[l]{//},
  morecomment=[s]{/*}{*/},
  morecomment=[s]{<!--}{-->},  
  commentstyle=\color{darkgray},
  stringstyle=\color{gray},
  morestring=[b]',
  morestring=[b]"
}
\lstset{breaklines=true}

% linewrap symbol
\usepackage{color}
\definecolor{grey}{RGB}{130,130,130}
\newcommand{\linewrap}{\raisebox{-.6ex}{\textcolor{grey}{$\hookleftarrow$}}}

% todo macro
\usepackage{color}
\newcommand{\todo}[1]{\noindent\textcolor{red}{{\bf \{TODO}: #1{\bf \}}}}

\def\JSONLD{\mbox{JSON-LD}}

\hyphenation{WebVTT}

\def\JSONLD{\mbox{JSON-LD}}

\begin{document}

\mainmatter  % start of an individual contribution

% first the title is needed
\title{Self-Contained Hypervideos\\ Using Web Components}

% a short form should be given in case it is too long for the running head
\titlerunning{Self-Contained Hypervideos Using Web Components}

% the name(s) of the author(s) follow(s) next
\author{
  Thomas Steiner\textsuperscript{1} \and
  Pierre-Antoine Champin\textsuperscript{1} \and \\
  Benoît Encelle\textsuperscript{1}\and
  Yannick Prié\textsuperscript{2}
}
%
\authorrunning{Steiner, Champin, Encelle, and Prié}
% (feature abused for this document to repeat the title also on left hand pages)

% the affiliations are given next
\institute{
  \textsuperscript{1}CNRS, Université de Lyon, LIRIS -- UMR5205, Université Lyon~1, France\\
  \email{\{tsteiner, pierre-antoine.champin\}@liris.cnrs.fr, benoit.encelle@univ-lyon1.fr}\\
  \textsuperscript{2}CNRS, Université de Nantes, LINA -- UMR 6241, France\\
  \email{yannick.prie@univ-nantes.fr}
}

\maketitle

\begin{abstract}
The creation of hypervideos is a~manual and tedious task,
requiring the preparation of assets like still frames,
the segmentation of videos in scenes or chapters,
and sometimes even the isolation of objects within the video.
In this paper, we propose a~semi-automated
Web-Components-based approach to self-contained hypervideo creation.
By \emph{self-contained} we mean that all necessary intrinsic components
of the hypervideo, for example, still frames should come from the video itself
rather than be included as external assets.
By leveraging the evolving Web Components standard,
we obtain a~high degree of abstraction,
which reduces the task of creating hypervideos
to textually marking them up with custom HTML elements.

\keywords{Hypervideo, Web Components}
\end{abstract}

\section{Introduction}

The term \emph{hypervideo} is commonly used to refer to
\textit{``a~displayed video stream that contains embedded user-clickable anchors''}%
~\cite{sawhney1996hypercafe,smith2002extensible},
allowing for navigation between the video and other hypermedia elements.
In a~2006 article in \emph{The Economist}, the authors write 
\textit{``[h]yperlinking video involves the use of ``object-tracking'' software
to make filmed objects, such as cars, clickable as they move around.
Viewers can then click on items of interest in a~video
to watch a related clip; after it has played,
the original video resumes where it left off.
To inform viewers that a~video is hyperlinked,
editors can add highlights to moving images, use beeps as audible cues,
or display still images from hyperlinked videos
next to the clip that is currently playing''}~\cite{economist2006hypervideo}.
In standard literature, hypervideo is considered a~logical consequence
of the related concept of \emph{hypertext}.
In contrast to hypertext, hypervideo necessarily includes a~time component,
as content changes over time.
In consequence, hypervideo has different technical and aesthetic requirements
than hypertext, the most obvious one being appropriate segmentation in scenes
or even objects.

\bibliographystyle{abbrv}
\bibliography{references}
\end{document}
