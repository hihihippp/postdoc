\documentclass{sig-alternate}

\usepackage[utf8]{inputenc}
\usepackage{enumitem}
\usepackage[hyphens]{url}
\usepackage[pdftex,urlcolor=black,colorlinks=true,linkcolor=black,citecolor=black]{hyperref}
\def\sectionautorefname{Section}
\def\subsectionautorefname{Subsection}

\usepackage{CJKutf8}

\newcommand{\superscript}[1]{\ensuremath{^{\textrm{#1}}}}

% listings and Verbatim environment
\usepackage[usenames,dvipsnames,svgnames,table]{xcolor}
\usepackage{fancyvrb}
\usepackage{relsize}
\usepackage{listings}
\usepackage{verbatim}
\newcommand{\defaultlistingsize}{\fontsize{8pt}{9.5pt}}
\newcommand{\inlinelistingsize}{\fontsize{8pt}{11pt}}
\newcommand{\smalllistingsize}{\fontsize{7.5pt}{9.5pt}}
\newcommand{\listingsize}{\defaultlistingsize}
\RecustomVerbatimCommand{\Verb}{Verb}{fontsize=\inlinelistingsize}
\RecustomVerbatimEnvironment{Verbatim}{Verbatim}{fontsize=\defaultlistingsize}
\lstset{frame=lines,captionpos=b,numberbychapter=false,escapechar=§,
        aboveskip=2em,belowskip=1em,abovecaptionskip=0.5em,belowcaptionskip=0.5em,
        framexbottommargin=-1em,basicstyle=\ttfamily\listingsize\selectfont}

% use Courier from this point onward
\let\oldttdefault\ttdefault
\renewcommand{\ttdefault}{pcr}
\let\oldurl\url
\renewcommand{\url}[1]{\inlinelistingsize\oldurl{#1}}

\lstdefinelanguage{JavaScript}{
  keywords={console, log, addEventListener, onmessage, alert, push, typeof, new, true, false, catch, function, return, null, catch, switch, var, if, in, while, do, else, case, break},
  keywordstyle=\bfseries,
  ndkeywords={class, export, boolean, throw, implements, import, this},
  ndkeywordstyle=\color{darkgray}\bfseries,
  identifierstyle=\color{Maroon},
  sensitive=false,
  comment=[l]{//},
  morecomment=[s]{/*}{*/},
  commentstyle=\color{ForestGreen},
  stringstyle=\color{Blue},
  morestring=[b]',
  morestring=[b]"
}

% linewrap symbol
\usepackage{color}
\definecolor{grey}{RGB}{130,130,130}
\newcommand{\linewrap}{\raisebox{-.6ex}{\textcolor{grey}{$\hookleftarrow$}}}

% todo macro
\usepackage{color}
\newcommand{\todo}[1]{\noindent\textcolor{red}{{\bf \{TODO}: #1{\bf \}}}}

\begin{document}
%
% --- Author Metadata here ---
\conferenceinfo{International Conference on Multimedia Retrieval}{2014 Glasgow, UK}
\CopyrightYear{2014} % Allows default copyright year (20XX) to be over-ridden - IF NEED BE.
%\crdata{0-12345-67-8/90/01}  % Allows default copyright data (0-89791-88-6/97/05) to be over-ridden - IF NEED BE.
% --- End of Author Metadata ---

\title{Telling Breaking News Stories from Wikipedia with Social Multimedia: A~Case Study of the 2014 Winter Olympics}

\numberofauthors{1}

\author{
% 1st. author
\alignauthor
Jon Doe\titlenote{Anonymized double blind submission in accordance with the ICMR2014 guidelines}\\%Thomas Steiner\titlenote{Thomas Steiner's second affiliation is \emph{Université de Lyon, CNRS Université Lyon~1, LIRIS, UMR5205, F-69622}}\\
  \affaddr{Random University}\\%\affaddr{Google Germany GmbH}\\\affaddr{Google Germany GmbH}\\
  \affaddr{Random Street 0, 12345 Random City}\\       %\affaddr{ABC-Str.~19, 20354 Hamburg, Germany}\\
       \email{jon.doe@example.org}%\email{tomac@google.com}
}

\maketitle
\begin{abstract}
With the ability to watch Wikipedia
and Wikidata edits in realtime,
the online encyclopedia and the knowledge base
have become increasingly used targets of research
for the detection of breaking news events.
In this paper, we present a~case study of the
\emph{2014 Winter Olympics}, where we tell the story of
breaking news events in the context of the Olympics
with the help of social multimedia
stemming from multiple social networks.
Therefore, we have extended the application
\emph{Wikipedia Live Monitor}---%
a~tool for the detection of breaking news events---%
with the capability of automatically creating
media galleries that illustrate events.
Athletes winning an Olympic competition,
a~new country leading the medal table,
or simply the Olympics themselves are all events
newsworthy enough for people to concurrently
edit Wikipedia and Wikidata---%
around the world in many languages.
The Olympics being an event of common interest,
an even bigger majority of people share the event
in a~multitude of languages on global social networks.
This sharing of moments in the form of comments and multimedia
happens either in people's role as spectators or participants
directly at one of the Olympic sites,
or as indirect second screen users
in front of their TV~sets at home.
With this work, we connect the world of
Wikipedia and Wikidata with the world of social networks,
in order to convey the spirit of the
\emph{2014 Winter Olympics},
to tell the story of victory and defeat,
and following the Olympic motto \emph{Citius, Altius, Fortius}.

\end{abstract}

\category{H.5.1}{Information Interfaces and Presentation}{Multimedia Information Systems}

%\terms{Human Factors, Languages, Measurement, Experimentation}

\keywords{Storytelling, social networks, multimedia, Wikipedia}

\section{Introduction}

\subsection{Brief History of Wikipedia and Wikidata}

\paragraph{The Online Encyclopedia Wikipedia}

The free online encyclopedia Wikipedia%
\footnote{Wikipedia: \url{http://www.wikipedia.org/}}~\cite{sanger05historywikipedia} was formally launched
on January 15, 2001 by Jimmy Wales
and Larry Sanger,
albeit the fundamental wiki technology
and the underlying concepts are older.
Wikipedia's direct predecessor was Nupedia~%
\cite{sanger05historywikipedia},
a~similarly free online encyclopedia,
however, that was exclusively edited by experts
following a~strict peer-review process.
Wikipedia's initial role was to serve
as a~collaborative platform for draft articles for Nupedia.
What happened in practice was that Wikipedia
rapidly overtook Nupedia as there was no peer-review burden
and it is now a~globally successful Web encyclopedia
available in 287 languages with overall
more than 30 million articles.%
\footnote{Wikipedia statistics: \url{http://stats.wikimedia.org/}}

\paragraph{The Knowledge Base Wikidata}

Wikidata\footnote{Wikidata: \url{http://www.wikidata.org/}}~\cite{vrandecic2012wikidata}
is a~free knowledge base that can be read
and edited by both humans and bots.
As Wikipedia is a~truly global effort,
sharing non-language-dependent facts
like population figures centrally
in a~knowledge base makes a~lot of sense
to facilitate international article expansion.
The knowledge base centralizes access to
and management of structured data,
such as references between Wikipedias
and statistical information that can be used in articles.
Controversial facts such as borders in conflict regions
can be added with multiple values and sources,
so that Wikipedia articles can,
dependent on their standpoint, choose preferred values.

\subsection{Social Network Sites and Multimedia}

In~\cite{boyd2007socialnetworksites},
boyd and Ellison define the term
\emph{social network site} as follows.
\textit{``We define social network sites as web-based services
that allow individuals to
\emph{(1)}~construct a~public or
semi-public profile within a~bounded system,
\emph{(2)}~articulate a~list of other users
with whom they share a~connection, and
\emph{(3)}~view and traverse their list of connections
and those made by others within the system.
The nature and nomenclature of these connections
may vary from site to site.''}
Social network sites commonly allow their users
to publish, share, and react or comment on social multimedia files
like videos or photos.
Mobile devices like smartphones or tablets
are omnipresent at all sorts of events,
enabling broad multimedia coverage.

\subsection{Hypotheses and Research Questions}

In this paper, we connect the world of
Wikipedia and Wikidata with the world of social networks
in order to convey the spirit of the
\emph{2014 Winter Olympics}.
We automatically generate media galleries of different types
for breaking news events around the Olympics
and evaluate the media galleries' relevance
and their visual aesthetics.
While this paper presents a~case study
of the \emph{2014 Winter Olympics},
the overall objective is to make the system domain-inde\-pendent.
We are steered by the following hypotheses.

\begin{itemize}
  \itemsep0em
  \item[(H1)] Social multimedia is suitable for illustrating
    breaking news events around the \emph{2014 Winter Olympics}.
  \item[(H2)] Given different types of media galleries
    for the same breaking news event
    around the \emph{2014 Winter Olympics},
    there is always one predictable preferred type.
  \item[(H2)] The key learnings from the domain of the
    \emph{2014 Winter Olympics} can be generalized
    to other domains.    
\end{itemize}

\noindent These hypotheses lead us to the research questions below.

\begin{itemize}
  \itemsep0em
  \item[(Q1)] What breaking news event features
    determine the relevancy of the corresponding media gallery?
  \item[(Q2)] What factors determine the choice
    of the preferred media gallery type for a~breaking news event?
\end{itemize}

The remainder of this paper is structured as follows.
\todo{Describe paper structure}

\cite{steiner2013mjnomore}
\cite{steiner2013tocrop}
\cite{steiner2013meteoroid}
\cite{steiner2013clustering}
\cite{steiner2013bots}
\cite{steiner2012aesthetic}

\bibliographystyle{abbrv}
\bibliography{references}
\balancecolumns
\end{document}