\documentclass{sig-alternate}

\usepackage[utf8]{inputenc}
\usepackage[hyphens]{url}
\usepackage[pdftex,urlcolor=black,colorlinks=true,linkcolor=black,citecolor=black]{hyperref}

% listings and Verbatim environment
\usepackage{fancyvrb}
\usepackage{relsize}
\usepackage{listings}
\usepackage{verbatim}
\newcommand{\defaultlistingsize}{\fontsize{8pt}{9.5pt}}
\newcommand{\inlinelistingsize}{\fontsize{8pt}{11pt}}
\newcommand{\smalllistingsize}{\fontsize{7.5pt}{9.5pt}}
\newcommand{\listingsize}{\defaultlistingsize}
\RecustomVerbatimCommand{\Verb}{Verb}{fontsize=\inlinelistingsize}
\RecustomVerbatimEnvironment{Verbatim}{Verbatim}{fontsize=\defaultlistingsize}
\lstset{frame=lines,captionpos=b,numberbychapter=false,escapechar=§,
        aboveskip=2em,belowskip=1em,abovecaptionskip=0.5em,belowcaptionskip=0.5em,
        framexbottommargin=-1em,basicstyle=\ttfamily\listingsize\selectfont}

% use Courier from this point onward
\let\oldttdefault\ttdefault
\renewcommand{\ttdefault}{pcr}
\let\oldurl\url
\renewcommand{\url}[1]{\inlinelistingsize\oldurl{#1}}

\lstdefinelanguage{JavaScript}{
  keywords={push, typeof, new, true, false, catch, function, return, null, catch, switch, var, if, in, while, do, else, case, break},
  keywordstyle=\bfseries,
  ndkeywords={class, export, boolean, throw, implements, import, this},
  ndkeywordstyle=\color{darkgray}\bfseries,
  identifierstyle=\color{black},
  sensitive=false,
  comment=[l]{//},
  morecomment=[s]{/*}{*/},
  commentstyle=\color{darkgray},
  stringstyle=\color{red},
  morestring=[b]',
  morestring=[b]"
}

% linewrap symbol
\usepackage{color}
\definecolor{grey}{RGB}{130,130,130}
\newcommand{\linewrap}{\raisebox{-.6ex}{\textcolor{grey}{$\hookleftarrow$}}}

\begin{document}
%
% --- Author Metadata here ---
\conferenceinfo{International World Wide Web Conference}{2014 Seoul, Korea}
\CopyrightYear{2014} % Allows default copyright year (20XX) to be over-ridden - IF NEED BE.
%\crdata{0-12345-67-8/90/01}  % Allows default copyright data (0-89791-88-6/97/05) to be over-ridden - IF NEED BE.
% --- End of Author Metadata ---

\title{Read/Write Online Semantic Video Annotation\\
by Leveraging WebVTT and JSON-LD}

\numberofauthors{6}

\author{
% 1st. author
\alignauthor
Thomas Steiner\titlenote{Thomas Steiner's second affiliation is \emph{T.~Steiner, Google Germany GmbH, ABC-Str.~19, 20354 Hamburg, Germany}}\\
       \affaddr{Université de Lyon} 
       \affaddr{CNRS Université Lyon~1}\\
       \affaddr{LIRIS, UMR5205}\\
       \affaddr{F-69622 France}\\
       \email{tsteiner@liris.cnrs.fr}
% 2nd. author
\alignauthor
Pierre-Antoine Champin\\
       \affaddr{Université de Lyon} 
       \affaddr{CNRS Université Lyon~1}\\
       \affaddr{LIRIS, UMR5205}\\
       \affaddr{F-69622 France}\\
       \email{champin@liris.cnrs.fr}
% 3rd. author
\alignauthor
Benoît Encelle\\
       \affaddr{Université de Lyon} 
       \affaddr{CNRS Université Lyon~1}\\
       \affaddr{LIRIS, UMR5205}\\
       \affaddr{F-69622 France}\\
       \email{encelle@liris.cnrs.fr}
\and  % use '\and' if you need 'another row' of author names
% 4th. author
\alignauthor
Yannick Prié\\
       \affaddr{Université de Nantes}\\
       \affaddr{LINA - UMR 6241}\\
       \affaddr{CNRS, France}\\
       \email{yannick.prie@univ-nantes.fr}
% 5th. author
\alignauthor
Ruben Verborgh\\
       \affaddr{Ghent University}\\
       \affaddr{iMinds -- Multimedia Lab}\\
       \affaddr{G.~Crommenlaan~8 bus~201}\\
       \affaddr{9050~Ghent, Belgium}\\
       \email{ruben.verborgh@ugent.be}
% 6th. author
\alignauthor
Rik Van de Walle\\
       \affaddr{Ghent University}\\
       \affaddr{iMinds -- Multimedia Lab}\\
       \affaddr{G.~Crommenlaan~8 bus~201}\\       
       \affaddr{9050~Ghent, Belgium}\\
       \email{rik.vandewalle@ugent.be}
}
% There's nothing stopping you putting the seventh, eighth, etc.
% author on the opening page (as the 'third row') but we ask,
% for aesthetic reasons that you place these 'additional authors'
% in the \additional authors block, viz.
\additionalauthors{Additional authors: John Smith (The Th{\o}rv{\"a}ld Group,
email: {\texttt{jsmith@affiliation.org}}) and Julius P.~Kumquat
(The Kumquat Consortium, email: {\texttt{jpkumquat@consortium.net}}).}
\date{30 July 1999}
% Just remember to make sure that the TOTAL number of authors
% is the number that will appear on the first page PLUS the
% number that will appear in the \additionalauthors section.

\maketitle
\begin{abstract}

\end{abstract}

\category{H.4}{Information Systems Applications}{Miscellaneous}
\category{D.2.8}{Software Engineering}{Metrics}[complexity measures, performance measures]

\terms{Theory}

\keywords{video annotation}

\section{Introduction}

\section{Related Work}

\cite{vandeursen2012mediafragmentannotations}

\section{Enabling Technologies}

\subsection{Web Video Text Tracks format (WebVTT)}

The Web Video Text Tracks format (WebVTT,~\cite{pfeiffer2013webvtt})
is intended for marking up external text track resources
for the purpose of captioning video content.
The recommended file extension is \texttt{vtt},
the MIME type is \texttt{text/vtt}.
WebVTT files are encoded in UTF-8 and
start with the required string \texttt{WEBVTT}.
Each file consists of items called \emph{cues}
that are separated by an empty line.
Each cue has a~start time and an end time in
\texttt{hh:mm:ss.milliseconds} format,
separated by a~stylized ASCII arrow \texttt{-}\texttt{->}.
The cue payload follows in the line after the cue timings part
and can span multiple lines.
Typically, the cue payload contains text,
but can also contain textual data serialization formats like JSON.
Cues can optionally have unique IDs
in form of an identifier followed by a~colon in the line
above the cue timings part.
\autoref{listing:webvtt} shows an exemplary WebVTT file.

\begin{lstlisting}[caption={Exemplary WebVTT file%
~\cite{pfeiffer2013webvtt} with two cues
with the IDs \texttt{cue1} and \texttt{cue2} respectively},
  label=listing:webvtt, float=h!]
WEBVTT

cue1:
00:00:01.000 --> 00:00:04.000
Never drink liquid nitrogen.

cue2:
00:00:05.000 --> 00:00:09.000
- It will perforate your stomach.
- You could die.
\end{lstlisting}

Web browsers support five different kinds of WebVTT text tracks,
listed in \autoref{table:texttrackkinds}

\begin{table*}[h]\footnotesize
\begin{tabular}{ r p{15cm} }
\textbf{WebVTT Kind} & \textbf{Description}\\

subtitles & Transcription or translation of the dialogue,
suitable for when the sound is available but not understood
(\emph{e.g.}, because the user does not understand the language). Overlaid on the video.\\

captions & Transcription or translation of the dialogue,
sound effects, relevant musical cues,
and other relevant audio information,
suitable for when sound is unavailable or not clearly audible
(\emph{e.g.}, because the user is deaf). Overlaid on the video;
labeled as appropriate for the hard-of-hearing.\\

descriptions & Textual descriptions of the video component
of the media resource, intended for audio synthesis
when the visual component is obscured, unavailable, or not usable
(\emph{e.g.}, because the user is interacting with the application
without a screen while driving, or because the user is blind).
Synthesized as audio.\\

chapters & Chapter titles, intended to be used for navigating
the media resource. Displayed as an interactive (potentially nested)
list in the user agent's interface.\\

metadata & Tracks intended for use from script.
Not displayed by the user agent.\\
\end{tabular}
  \caption{Different WebVTT text track kinds}
  \label{table:texttrackkinds}
\end{table*}

\section{Implementation}

\section{Evaluation and Discussion}

\section{Future Work and Conclusions}

\section*{Acknowledgments}

The research presented in the context of this work
was partially supported by the ANR project
\emph{Spectacle En Ligne(s)}, project reference
\mbox{ANR-12-CORP-0015}.

\bibliographystyle{abbrv}
\bibliography{references}

\end{document}